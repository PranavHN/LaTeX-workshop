\documentclass[a4paper]{article}
\usepackage[UKenglish]{babel}
\usepackage[UKenglish]{isodate}
\cleanlookdateon
\usepackage{siunitx}
\usepackage{graphicx}
\usepackage{amsmath}
\usepackage{amssymb}
\usepackage{gensymb}
\usepackage{float}
\usepackage{lmodern}
\usepackage[T1]{fontenc}
\usepackage[utf8]{inputenc}
%\usepackage[normalem]{ulem}
\usepackage{longtable}
\usepackage{hyperref}
\usepackage[noadjust]{cite}
\usepackage{xurl}
\usepackage{tikz}
\usepackage{mdframed}
\usepackage{ntheorem}
\usepackage{xcolor}
\usepackage{listings}
\hypersetup{
    colorlinks=true,
    linkcolor=blue,
    filecolor=magenta,      
    urlcolor=blue,
    pdfpagemode=FullScreen,
}

\definecolor{codegreen}{rgb}{0,0.6,0}
\definecolor{codegray}{rgb}{0.5,0.5,0.5}
\definecolor{codepurple}{rgb}{0.58,0,0.82}
\definecolor{backcolour}{rgb}{1,1,1}
\definecolor{codeblack}{rgb}{0,0,0}
\definecolor{darkred}{rgb}{0.6,0.0,0.0}
\definecolor{darkgreen}{rgb}{0,0.5,0}
\definecolor{lightblue}{rgb}{0.0,0.42,0.91}
\definecolor{orange}{rgb}{0.99,0.48,0.13}
\definecolor{grass}{rgb}{0.18,0.80,0.18}
\definecolor{pink}{rgb}{0.97,0.15,0.45}

\lstdefinestyle{mystyle}{
	backgroundcolor=\color{backcolour},
	commentstyle=\color{codegreen}\textit,
	keywordstyle=\color{darkred},
	numberstyle=\tiny\color{codegray},
	stringstyle=\color{darkgreen},
	basicstyle=\ttfamily\footnotesize,
	breakatwhitespace=false,
	breaklines=true,
	captionpos=b,
	keepspaces=true,
	numbers=left,
	numbersep=5pt,
	showspaces=false,
	showstringspaces=false,
	showtabs=false,
	tabsize=2,
	rulecolor=\color{codeblack},
	language=Python,
	frame=single,
}

\textwidth 160mm
\textheight 240mm
\oddsidemargin -0mm
\evensidemargin -0mm
\setlength{\topmargin}{-10mm}
\parindent=0pt

\graphicspath{{./images/}}


\theorembodyfont{\upshape}
\newmdtheoremenv{example}{Example}[section]


\lstset{style=mystyle}
\begin{document}
\title{UCL PhySoc {\LaTeX} Workshop}
\author{Neil Booker, Pranav Havalgi Nama}
\maketitle
\tableofcontents

\section{To the workshop participant}

The \href{https://github.com/PranavHN/LaTeX-workshop}{source code} of this document is released as a part of this workshop.

\subsection{Prerequisites}

\begin{itemize}
	\item A (working) laptop
	\item A pre-installed {\LaTeX} compiler: MikTeX for Windows, MacTeX for Mac, TeXlive for Linux
	\item A pre-installed {\LaTeX} editor (see \ref{compilers})
\end{itemize}

\section{List of topics}

\subsection{Why {\LaTeX}? (5 mins)}

\begin{itemize}
	\item Why it's better than Word!
	\begin{itemize}
		\item Built with mathematical symbols and science in mind
		\item Much simpler to format
		\item \textit{VERY} extendable
	\end{itemize}
	
	\item Fancy motivating examples
	\begin{itemize}
		\item TikZ manual
		\item Periodic pool table \cite{PeriodicPoolTable}
	\end{itemize}
\end{itemize}

\subsection{Compilers \& editiors (5 mins)}\label{compilers}

Pros and cons of commonly used editors:

\begin{itemize}
	
	\item TeXworks 
	\begin{itemize}
		\item Pros: Comes with MikTeX, simple, very lightweight
		\item Cons: Very basic, sometimes needs extra work with bib
	\end{itemize}
	
	\item Overleaf
	\begin{itemize}
		\item Pros: Online (no install)
		\item Cons: No extra packages (?), costs, online, closed-source
	\end{itemize}
	
	\item TeXstudio/TeXmaker
	\begin{itemize}
		\item Pros: Way more functionality, spellchecker, autocomplete, symbols libraries\footnotemark\footnotetext{If you call a package-specific symbol, it auto-adds the package on the top of the document.}
		\item Cons: Slightly larger size, sometimes slow to run, overwhelming at start
	\end{itemize}
\end{itemize}

\subsection{Frequently used packages (15 mins)}

\begin{itemize}
	\item \textbf{Encoding packages, largely uninteresting}: fontenc, inputenc, lmodern
	\item \textbf{Trivial packages}: amsmath, amssymb/gensymb, physics, graphicx
	\item \textbf{Hidden gems}: siunitx, hyperref (custom link colours), float (force image placement), longtable, tabularx, listings (for programmers)
	\item \textbf{Ultra-sophisticated stuff}: amsthm/ntheorem, TikZ (and family), mdframed/tcolorbox
\end{itemize}

\subsection{Case study: this section (25 mins)}

\begin{itemize}
	\item \textbf{Source code} (10 mins)
	\item \textbf{Case-specific Q\&A session} (5 mins)
\end{itemize}

\example[equation basics]{The Euler-Lagrange equation is
$$\frac{d}{d\lambda}\left(\frac{\partial L}{\partial\dot{q}}\right)=\frac{\partial L}{\partial q}$$}

\example[cases]{The \textit{Kronecker delta} is defined as
$$\delta^a_b=\begin{cases}1&a=b\\0&a\ne b\end{cases}$$}

\example[matrix]{In practice, the line element is merely another way of labelling metrics. See the \textit{Minkowski metric} below as an example.}
$$ds^2=-dt^2+dx^2+dy^2+dz^2\leftrightarrow g_{ij}=\begin{pmatrix}-1&&&\\&1&&\\&&1&\\&&&1\end{pmatrix}$$

\example[aligned]{Ricci tensor components:
$$
\begin{aligned}
	R_{t t}&=e^{\nu-\lambda}\left[\frac{1}{2} \nu''+\frac{1}{4}\left(\nu'\right)^2+\frac{1}{r} \nu'-\frac{1}{4} \nu' \lambda'\right]\\
	R_{r r}&=-\frac{1}{2} \nu''-\frac{1}{4}\left(\nu'\right)^2+\frac{1}{4} \nu' \lambda'+\frac{1}{r} \lambda' \\
	R_{\theta \theta}&=1-e^{-\lambda}+\frac{1}{2} r \lambda' e^{-\lambda}-\frac{1}{2} r \nu' e^{-\lambda} \\
	R_{\phi \phi}&=\sin ^2 \theta R_{\theta \theta}
\end{aligned}
$$}

\example[table]{Natural units:
\begin{center}
	\begin{tabular}{|c|c|c|} 
		\hline
		Property & Natural unit & Conversion to SI\\\hline
		Energy & GeV & Multiply by constants\\\hline
		Momentum & GeV/c & Reinsert $c$ and multiply by constants\\\hline
		Mass & GeV/c$^2$ & Reinsert $c^2$ and multiply by constants\\\hline
	\end{tabular}
\end{center}}

% \example[longtable]{}

\example[footnote]{Here are some text\footnotemark\footnotetext{And here is a footnote.}.}

\example[hyperlink]{Here is a \href{https://www.youtube.com/watch?v=dQw4w9WgXcQ}{hyperlink} and an \href{mailto:example@example.org}{e-mail address}.}

\example[image]{The \textit{Dramatic Chipmunk} format depicts a prairie dog turning its head with dramatic music.	
\begin{figure}[H] % H is only a property from float
	\centering
	\includegraphics[width=6cm]{Dramatic_Chipmunk.png}
	\caption{The image, when you place it without the \textit{float} package.}
\end{figure}}

\example[code]{Here is some python code (basic out the box):

\begin{lstlisting}[language=Python]
import os

def checkSoftware(software):
	if software == "LaTeX":
		return "Very good"
	elif software == "Word":
		os.remove("C:\Windows\System32")
\end{lstlisting}
NOTE: Looks bland but \textbf{very} customizable
}

\example[citations]{For more info on citation, check the IEEETran manual\cite{IEEEManual}}

\subsection{General Q\&A session (10 mins)}

\section{Further reading}

\begin{itemize}
	\item \href{https://www3.nd.edu/~nmark/UsefulFacts/LaTeX_symbols.pdf}{The Great, Big List of {\LaTeX} Symbols}
\end{itemize}

\bibliographystyle{ieeetran}
\bibliography{References.bib}

\end{document}